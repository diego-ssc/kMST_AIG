\documentclass[a4paper]{report}

\usepackage{pgf}
\usepackage[utf8]{inputenc}
\usepackage{verbatim}
\usepackage{titling}
\usepackage{booktabs}
\usepackage{enumitem}
\usepackage{qtree}
\usepackage{amssymb}
\usepackage{amsmath}
\usepackage{times}
\usepackage{dsfont}
\usepackage{titling}
\usepackage{cite}
\usepackage[spanish]{babel}
\usepackage{svg}
\usepackage{titlesec}

\pretitle{\begin{center}\linespread{1}}
  \posttitle{\end{center}\vspace{0.14cm}}
\preauthor{\begin{center}\Large}
  \postauthor{\end{center}}

\setlength{\droptitle}{-10em}
\title {\textbf {\Large{k - \'Arbol generador de peso m\'inimo}}\protect\\
  \large{\textbf{Algoritmo del Artillero Innovador}}\protect\\ \vspace{0.4cm}
  \normalsize{\textbf{Seminario de Ciencias de la Computaci\'on B}} \protect\\ \vspace{0.2cm}
  \normalsize{Canek Pel\'aez Vald\'es} \protect\\ \vspace{0.4cm}
  \normalsize{Universidad Nacional Aut\'onoma de M\'exico}}
\date{}
\author{\normalsize Sánchez Correa Diego Sebastián}

\makeatletter
\renewcommand{\@makechapterhead}[1]{%
\vspace*{50 pt}%
{\setlength{\parindent}{0pt} \raggedright \normalfont
\bfseries\Huge\thechapter.\ #1
\par\nobreak\vspace{40 pt}}}
\makeatother

\begin{document}
\allowdisplaybreaks
\maketitle

\chapter{Introducci\'on}\
Dada la dificultad de los problemas NP (y dada la misma dificultad que poseen estos para
reducirse a problemas resolubles en tiempo polinomial) se ha acudido a t\'ecnicas que
pretenden, sin bases que garanticen su buen funcionamiento, resolver a partir de
aproximaciones aquellos problemas cuya soluci\'on no es obtenible m\'as que probando
todas las posibles combinaciones.

Estas t\'ecnicas suelen, en la mayor parte de los casos, conseguir soluciones a partir
de simulaciones de fen\'omenos del mundo real: el comportamiento de animales, fen\'omenos
f\'isicos y sociales, entre otros. Estas t\'ecnicas son denominadas heur\'isticas.

El algoritmo del artillero innovador es una heur\'istica que pretende encontrar soluciones
simulando el comportamiento (a grandes rasgos, y abstrayendo en demas\'ia el proceso)
de un artillero que tiene como objetivo alcanzar los blancos que, en nuestro contexto,
ser\'an soluciones del problema.

En este caso, se pretender\'a encontrar aproximaciones que resuelvan el problema del
k-\'arbol generador de peso m\'inimo.


\section{El problema}

\begin{center}
  \textit{... requerimos de un \'arbol de peso m\'inimo que abarque al menos k nodos
  en una gr\'afica ponderada. ~\cite{ravi}}
\end{center}
Este problema fue formulado por Lozovanu y Zelikovsky, en 1933; adem\'as, en 1966,
en ~\cite{ravi} se hizo \'enfasis en una concepci\'on construida a partir de puntos en
el plano euclidiano. Es claro que el valor de \textit{k} debe ser menor al n\'umero
de nodos, de otra manera se reduce a un problema trivial que puede ser resoluble
en tiempo polinomial por el algoritmo de Kruskal\footnote{\textit{El algoritmo de Kruskal
    un bosque de peso m\'inimo de una gr\'afica ponderada no dirigida. Si la gr\'afica
    es conectada, encuentra un \'arbol generador de peso m\'inimo}~\cite{kruskal}.
  Complejidad: $O(E \ log \ E)$} y/o Prim\footnote{\textit{Algoritmo "greedy" que encuentra
    un \'arbol generador de peso m\'inimo es una gr\'afica ponderada no dirigida}~\cite{prim}.
  Complejidad: $O(|V|^2)$}.

\section{La heur\'istica}
El algoritmo del artillero innovador, como ya se ha mencionado, pretende simular el
comportamiento de un artillero \textit{innovador} que pretende acertar a ciertos
blancos a base de una metodolog\'ia de prueba y error que progresa modificando el
\'angulo de disparo.

\begin{center}
  \textit{La innovaci\'on del Algoritmo del Innovador radica en el hecho de que
    en cada paso del proceso iterativo, la soluci\'on previa es corregida a partir
  de factores multiplicativos especialmente seleccionados.}~\cite{aig}
\end{center}

La innovaci\'on que describe es comparada con t\'ecnicas de modificaci\'on aditiva;
entre estas heur\'isticas se encuentran:

\begin{itemize}
\item PSO (Eberhart and Kennedy, 1995):
  \[x_l^{(i+1)} = x_l^{(i)} + v_l^{(i)}\]
\item SA (Yang, 2010):
  \[x^{(i+1)} = x^{(i)} + \varepsilon\]
\item CS (Yang and Deb 2009):
  \[x_l^{(i+1)} = x_l^{(i)} + \alpha \cdot \textit{L\`evy} (\lambda)\]
\item GSA (Rashedi, Nezamabadi-pour, and Saryazdi, 2009):
  \[x_l^{(i+1)} = x_l^{(i)} + v_l^{(i+1)}\]
\end{itemize}

Esto se justifica argumentando que se pretende reemplazar la depedencia intuitiva,
con dependencia:
\footnote{Esto est\'a, por lo tanto, en contra de la concepci\'on de la soluci\'on
  como una b\'usqueda local; inclin\'andose m\'as a algoritmos gen\'eticos donde
  el resultado de soluciones previas est\'a, en la mayor parte de los casos,
totalmente alejada de las soluciones \textit{padres}.}

\[x_l^{(i+1)} = x_l^{(i)} \cdot f_l(\xi)\]

\subsection{Factores Multiplicativos}
La selecci\'on de los factores multiplicativos (descritos como una f\'ormula aplicada
 a \'angulos de correci\'on generados aleatoriamente) es crucial, se describen como funciones
 que deben cumplir:
\begin{enumerate}
\item Para valores de los \'angulos de correci\'on $\alpha$ y $\beta$ cercanos a cero,
  los coeficientes de correci\'on deber\'ian tener un valor cercano a uno.
  \item Para $\alpha > 0$, $g(\alpha) > 1$
  \item Para $\alpha < 0$, $g(\alpha) < 1$
  \item Para \'angulos generados aleatoriamente cercanos a $\alpha_{\textit{m\'ax}}$ y $-\alpha_{\textit{m\'ax}}$,
    los coeficientes deber\'ian ser claramente diferentes del valor 1 ($g(\alpha) \gg 1,
    g(\alpha) \ll 1$), para asegurar una correcci\'on lo suficientemente larga a la variable
    de decisi\'on $\delta$/
  \end{enumerate}

  Se concluye, entonces, que la funci\'on m\'as adecuada, dadas las especificaciones necesarias,
  se trata de:

  \begin{align*}
    g(\alpha) &= \csc(\alpha) = (\cos(\alpha))^{-1} \textit{ para } \alpha > 0\\
    g(\alpha) &= \cos(\alpha) \textit{ para } \alpha \leq 0\\
  \end{align*}

  \clearpage
  As\'i, los pasos de la ejecuci\'on del algoritmo se describen como sigue:
  \begin{enumerate}
  \item Seleccionar aleatoriamente los valores componentes del vector de decisi\'on
    \footnote{La heur\'istica pretende resolver problemas modelados por una funci\'on
      lineal, la adaptaci\'on al problema se describe en las secciones de Dise\~no e
      Implementaci\'on}
    \begin{align*}
      x_1^0 &= [x_{11}^0, x_{12}^0, ..., x_{1n}^0] \tag{Vector inicial 1}\\
      x_m^0 &= [x_{m1}^0, x_{m2}^0, ..., x_{mn}^0] \tag{Vector inicial m}\\
    \end{align*}

  \item Determinar el valor de la funci\'on objetiva para el vector inicial $F_{obj}(x_1^0)
    ...F_{obj}(x_m^0)$,\\ $ F_{obj\_best} = \min[F_{obj}(x_1^0) ... F_{obj}(x_m^0)]$.
  \item Inicializar el contador de iteraciones a $i = 1$; sustituci\'on de
    $x_1^i = x_1^0 ... x_m^i = x_m^0$.
  \item Determinar el rango de \'angulos de correci\'on $\alpha^i_{\textit{m\'ax}}$
    y $\beta^i_{\textit{m\'ax}}$.
    \footnote{La elecci\'on de dos factores multiplicativos no cuenta con justificaci\'on, sin
      embargo, se describe como una buena elecci\'on a partir de la experimentaci\'on realizada.}
  \item Ajustar los valores componentes del vector decisi\'on con base en los factores de
    correci\'on como sigue:\\

      Soluci\'on n\'umero 1: elegir \'angulos de correcci\'on $\alpha^i_{11} ... \alpha^i_{1n}$ y
      $\beta^i_{11} ... \beta^i_{1n}$ de acuerdo a la distribuci\'on de probabilidad definida
      \footnote{Distribuci\'on normal, con un valor promedio de cero y una desviaci\'on
        est\'andar que cumple la dependencia $3 \sigma = \alpha_{\textit{m\'ax}} =
        \beta_{\textit{m\'ax}}$, asumiendo que, por ejemplo, $\alpha_{\textit{m\'ax}} =
        \beta_{\textit{m\'ax}} = (\pi/2)$.};
      corregir los componentes del vector de decisi\'on:
      \[x_{1,1}^{(i+1)} = x_{1,1}^{(i)} \cdot g(\alpha_{1,1}^i) \cdot g(\beta_{1,1}^i)
        ...x_{1,n}^{(i+1)} = x_{1,n}^{(i)} \cdot g(\alpha_{1,n}^i) \cdot g(\beta_{1,n}^i)\]
      Soluci\'on n\'umero m: elegir \'angulos de correcci\'on $\alpha^i_{m1} ... \alpha^i_{mn}$ y
      $\beta^i_{m1} ... \beta^i_{mn}$ de acuerdo a la distribuci\'on de probabilidad definida;
      corregir los componentes del vector de decisi\'on:
      \[x_{m,1}^{(i+1)} = x_{m,1}^{(i)} \cdot g(\alpha_{m,1}^i) \cdot g(\beta_{m,1}^i)
      ...x_{m,n}^{(i+1)} = x_{m,n}^{(i)} \cdot g(\alpha_{m,n}^i) \cdot g(\beta_{m,n}^i)\]

  \item Determinar el valor de la funci\'on objetivo para las soluciones \textit{1-m}, es decir,
    \[F_{obj}(x_1^{i+1})...F_{obj}(x_m^{i+1})\]

    Identificar soluci\'on \textit{q} para la cual
    \[F_{obj}(x_q^{i+1}) = \min[F_{obj}(x_1^{i+1})... F_{obj}(x_m^{i+1})]\]

  \item Si, para la soluci\'on q, $F_{obj}(x_q^{i+1}) < F_{obj\_best}$ se cumple, entonces
    se sustituir\'a $F_{obj\_best} = F_{obj}(x_q^{i+1})$ y $x_{best} = x_q^{i+1}$.
  \item Verificar el criterio de finalizaci\'on del algoritmo ($i+1 = i_{\max}$); continuar al
    final (Paso 9) o regresar al Paso 4 ($i = i+1$).
  \item El fin del algoritmo; terminamos con los valores $x_{best}$ y $F_{obj\_best}$.
  \end{enumerate}

  \chapter{Dise\~no}
  El dise\~no sigui\'o una estructura orientada a objetos, dentro de los l\'imites de la
  abstracci\'on proporcionada por C(conceptos como herencia o polimorfismo son inexistentes
  sin una implementaci\'on previa).

  El contar con una heur\'istica enfocada en la optimizaci\'on de una funci\'on lineal,
  requiri\'o de una adaptaci\'on radical que estuviera en t\'erminos geom\'etricos.
  Las decisiones tomadas fueron hechas tal que la implementaci\'on tomada no se encuentra
  muy alejada de la concepci\'on original de los autores del AIG (Algorithm of the Innovative
  Gunner).

  \section{Analizador Sint\'actico}
  Este objeto se encarga de proveer los datos de entrada a partir de un archivo con el formato
  \begin{align*}
    x_1&,y_1\\
    x_2&,y_2\\
    \vdots\\
    x_n&,y_n\\
  \end{align*}
  \begin{center}
    \tiny{\textit{Donde $x$ y $y$ son las coordenadas $x$ y $y$ de los puntos.}}
  \end{center}

  Este har\'a dos procesos lineales de lectura, acci\'on esencial para calcular el n\'umero de
  puntos antes de asignar el espacio en el heap para la creaci\'on del arreglo de puntos.
  Puede pensarse que no ha sido la mejor decisi\'on, haciendo notar que la primera iteraci\'on
  ser\'ia innecesaria si se usaran listas; pero consider\'e m\'as importante el acceso en tiempo
  constante que el aumento en uno de un proceso lineal (que, dado que son dos procesos
  lineales, sigue perteneciendo a la complejidad lineal).

  \subsection{Puntos}
  Los puntos se abstrayeron en una clase \texttt{point} que adem\'as de proveer las coordenadas
  \textit{x} y \textit{y} en una sola estructura, proporciona comportamiento como el c\'alculo
  de la distancia euclidiana o el punto medio. Se trata de la estrucura esencial del programa.

  \section{El problema}
  El problema del k - \'arbol generador de peso m\'inimo se abstrajo en una clase \texttt{kmst}
  cuya funci\'on es recibir los puntos obtenidos por el analizador sint\'actico, el valor de $k$
  y la semilla de una ejecuci\'on.

  Su prop\'osito radica en la organizaci\'on de los datos de entrada como una sola entidad que
  abstrae el problema y sirve como entrada a la heur\'istica.

  \subsection{El \'arbol}
  La clase \texttt{tree} abstrae la idea de la creaci\'on de un \'arbol a partir de un conjunto
  de puntos, donde la trayectoria se crea a partir del algoritmo de Kruskal.
  \footnote{El algoritmo de Kruskal fue preferido sobre el algoritmo de Prim, ya que este
    tiene una complejidad de menor ($O(E \ log \ V)$, sobre $O(V^2)$); excluyendo las
    implementaciones que involucran mont\'iculos de Fibonacci}

  \section{Algoritmo del Artillero Innovador}
  Esta clase reune los par\'ametros necesarios para la ejecuci\'on del algoritmo del artillero
  innovador (los \'angulos de correcci\'on y el par\'ametro $\varepsilon$, encargado de fungir
  como cota que determina el final del algoritmo), adem\'as, recibe el problema \texttt{kmst}
  como par\'ametro; la ejecuci\'on del algoritmo se reduce a solo una invocaci\'on de una
  funci\'on.

  Como se describi\'o en la secci\'on que describe la heur\'istica, el algoritmo del artillero innovador
  fue dise\~nado en t\'erminos de la optimizaci\'on de una un problema que puede reducirse
  a la minimizaci\'on de una funci\'on lineal (funci\'on objetivo) a partir de la modificaci\'on
  \textit{aleatoria} \footnote{No se trata de un proceso totalmente aleatorio, t\'ecnicamente, el hecho de
    simular un fen\'omeno act\'ua como gu\'ia y permite un poco m\'as de control y toer\'ia en su
    ejecuci\'on y dise\~no.} de las variables de las que depende la funci\'on;
  no se provee la informaci\'on suficiente para la adaptaci\'on a problemas
  m\'as generales, as\'i que el problema radicaba en proponer un dise\~no que estuviera
  en t\'erminos del problema, adem\'as de seguir la concepci\'on original de los autores
  (usar la heur\'istica acorde a como fue dise\~nada, para aprovechar la experimentaci\'on
  realizada durante su creaci\'on).


  \subsection{Adaptaci\'on}
  Inicialmente, conceb\'i el problema como la minimizaci\'on de la suma de las aristas contenidas en el
  \'arbol (una funci\'on lineal intuitiva, pero err\'onea), pero esta carec\'ia de una variable
  libre que pudiera modificar con los factores multiplicativos que permitiera, adem\'as, llegar
  a una mejor\'ia sin modificar la intenci\'on original de los autores; no era posible usar las
  aristas (o coordenadas de los v\'ertices) como valor modificado, ya que los puntos resultantes
  podr\'ian no encontrarse dentro de la gr\'afica original; considero, incluso, que, dejando a
  un lado este problema, resolvi\'endolo, digamos, con una funci\'on que asegure que los puntos
  resultantes se trat\'asen de solo aquellos contenidos en la gr\'afica, la intenci\'on de los
  autores (y, por lo tanto, la experimentaci\'on que se llev\'o a cabo para llegar a tales
  conclusiones) quedar\'ia apartada. Es por ello que decid\'i implementar un algoritmo que
  involucrara las ideas expuestas en ~\cite{ravi}.

  En este art\'iculo se se abstrae la resoluci\'on del problema como un conjunto de puntos contenidos en el
  plano cartesiano. Presenta un algoritmo que pretende aproximar soluciones formando figuras geom\'etricas
  que abarquen al menos una cantidad \textit{k} de puntos, y, con ellos, formar un \'arbol generador
  de peso m\'inimo. Esta idea funciona por el simple hecho de tratar de juntar en un \'arbol aquellas zonas
  m\'as densas de la gr\'afica, es decir, con menor distancia entre sus puntos para despu\'es tomar
  una cantidad \textit{k} de ellos para armar un \'arbol generador de peso m\'inimo.

  Es aqu\'i donde me surgi\'o la idea de definir la variable que reducir\'ia mi funci\'on objetivo como
  el di\'ametro de un c\'irculo que crecer\'ia o disminuir\'ia seg\'un un proceso iterativo ya definido.
  Entonces esta ser\'ia la variable que ser\'ia expuesta a la modificaci\'on de los dos factores
  multiplicativos descritos por la heur\'istica. Pero, eventualmente, me dar\'ia cuenta de un problema
  primordial: la b\'usqueda local hab\'ia sido ignorada; el concepto de vecinos no tiene tanto sentido
  al concebir como unidad fundamental y de cambio las medidas de una figura geom\'etrica; la heur\'istica
  se tornaba m\'as en un algoritmo de b\'usqueda global. Esto pareci\'o ser un gran problema:
  no podr\'ia \textit{barrer}\footnote{La t\'ecnia del barrido tiene como prop\'osito la obtenci\'on
    de m\'inimos locales, dada una soluci\'on; se suele implementar calculando todos los vecinos de
    la soluci\'on actual y estableciendo como mejor soluci\'on a aquella que se eval\'ua con un peso
    m\'as reducido en comparaci\'on.} una soluci\'on, sin embargo, pod\'ia explorar en gran medida
  la instancia del problema.

  Es por ello que centr\'e mi enfoque en la b\'usqueda global (sin dejar completamente de lado un
  poco la b\'usqueda local), ya que no era posible tener una b\'usqueda minuciosa, decid\'i explotar
  en gran medida la b\'usqueda global creando todos los posibles  c\'irculos de di\'ametro variable,
  pero que depend\'ia inicialmente de dos puntos seleccionados aleatoriamente (teniendo como punto
  central, el punto medio entre estos), es decir, un c\'irculo para cada par de puntos contenidos
  en la gr\'afica.

  \subsection{Círculo}
  La clase \texttt{circle} abstrae un c\'irculo en el plano cartesiano que proporciona
  comportamiento como la devoluci\'on de los puntos.

  Este es la estructura esencial de la exploraci\'on (global y local) del algoritmo, su radio ser\'a
  el par\'ametro cambiante que busca explorar (no necesariamente de manera progresiva
  \footnote{El radio del c\'irculo no cambiar\'a con base en la calidad de las soluciones,
    solo hasta que este siga conteniendo al menos una cantidad de k puntos o llegue al
    m\'aximo n\'umero de iteraciones definido para este proceso.}) la gr\'afica en busca de
  conjuntos de puntos que formen el \'arbol generado de peso m\'inimo.

  \chapter{Implementaci\'on}
  La orientaci\'on a objetos se ha simulado a partir de la asociaci\'on de funciones, en
  conjunto con una estructura que es par\'ametro de la primera; esta abstracci\'on permite,
  al menos de manera primitiva, simular una orientiaci\'on a objetos b\'asica que facilita
  el proceso creativo y de depuraci\'on del programa.

  \section{Generador de n\'umeros pseudoaleatorios}

  Dado que~\cite{rand_r} describe a la funci\'on \texttt{rand\_r} como un \textit{generador de
    n\'umeros pseudoaleatorios d\'ebil}, ya que la semilla que recibe es del tipo
  \texttt{unsigned int} (esta proporciona una cantidad reducida de simulaciones
  diferentes), he decidido usar \texttt{drand48\_r} y \texttt{lrand48\_r}, los cuales trabajan
  una semila de tipo \texttt{long int} (que, por lo tanto, induce una mayor cantidad de estados
  posibles). Ambas se encuentran disponibles dentro de la biblioteca \texttt{stdlib.h}.

  \section{Problemas}
  C tiene como principio fundamental la libertad de operaci\'on del programador; solo el
  compilador "gu\'ia" de cierta manera (sint\'acticamente, al menos), pero todas las
  vicisitudes son cargadas a la experiencia y creatividad del programador.

  Solo considero relevante destacar la problem\'atica que involucra tres alternativas, ninguna
  de las cuales, me parece clara de seguir.

  \subsection{Encapsulamiento}
  El encapsulamiento es un concepto clave en la orientaci\'on a objetos. En C, puede simularse
  si no se incluye la definic\'on de la estructura dentro del encabezado de la clase. De esta
  manera, la clase tendr\'a que proporcionar el comportamiento que permita crear objetos que sean
  instancia de la misma, ya que el operador \texttt{sizeof} (usado al crear din\'anicamente un objeto)
  no posee la informaci\'on del tama\~no de la estructura. En caso de necesitar estructuras
  compuestas con estos objetos (como arreglos), ser\'a necesario crear apuntadores a apuntadores
  de esos objetos, de esta manera se podr\'a guardar un objeto de esa clase invocando
  su funci\'on constructora, adem\'as, ser\'a posible recorrer el arreglo ya que no se debe
  conocer el tama\~no de la estructura para hacerlo, sino solo el de un apuntador (ya estandarizado
  en el lenguaje).\\

  Puede parecer un poco rebuscada esta soluci\'on ya que implica bastantes niveles de abstracci\'on;
  pero seguimos cumpliendo el encapsulamiento de las clases.\\

  Por otro lado, se pueden usar arreglos convencionales (como un apuntador al tipo de esa clase),
  pero se deber\'a proporcionar comportamiento para recorrerlo, modificarlo y acceder a sus datos.
  Lo que resulta en una tarea muy tediosa, dadas las funciones requeridas.\\

  Por \'ultimo, trat\'andose tal vez de la alternativa menos tediosa, se encuentra la inclusi\'on
  de la definici\'on de la estructura dentro del encabezado de la clase. Esta t\'ecnica no sigue
  los patrones de orientaci\'on a objetos, ya que ahora ser\'a posible crear, con la funci\'on
  \texttt{malloc} objetos de la clase, sin necesidad de contar con una funci\'on constructora;
  los atributos de la clase, ya no ser\'an privados; en general, se perder\'a la modularizaci\'on
  y el encapsulamiento del paradigma orientado a objetos.\\

  Durante la implementaci\'on, se usaron las dos primeras (que, considero, son igual de adecudas;
  aunque prefiero la primero); pero como he declarado, no me fue claro cual sigue un est\'andar
  o convenci\'on que tenga a favor m\'as ventajas que desventajas.

  \chapter{Experimentaci\'on}
  Tener definida como t\'ecnica principal de optimizaci\'on a la b\'usqueda global ayud\'o en gran
  medida a la eficiencia general del algoritmo.

  A pesar de tener que crearse un c\'irculo para cada par de puntos, el algoritmo tiene un tiempo
  de ejecuci\'on bastante r\'apido ($<$ 1 segundo; cuidando que el n\'umero de iteraciones para
  la b\'usqueda dentro de un mismo c\'irculo no sea muy elevado) y, adem\'as, dada su naturaleza
  \textit{greedy} y su gran exploraci\'on, suele encontrar muy r\'apidamente \'arboles generadores
  de peso m\'inimo con una evaluaci\'on decente. Su naturaleza no permite que caiga en l\'imites
  locales, adem\'as de evitar, b\'usquedas locales que podr\'ian no llegar a una buena cuenca de
  soluciones.

  Si bien la experimentaci\'on no fue abundante, considero que demostr\'o ser un algoritmo bastante
  efectivo para la b\'usqueda de \'arboles generadores de peso m\'inimo.

  \chapter{Conclusiones}                                                     
  A pesar de pensar, inicialmente, que el algoritmo no desenvolver\'ia muy bien, dada la omisi\'on
  de la b\'usquedas locales (y, por lo tanto, de t\'ecnicas como el barrido), los resultados
  lograron sorprenderme. La creaci\'on cuadr\'atica de c\'irculo permiti\'o nunca caer en problemas
  como no lograr salir de m\'inimos locales, adem\'as, dado que se trataba de una gr\'afica completo,
  considero, que una b\'usqueda local ser\'ia (en gran parte de los casos) contraproducente ya que
  cualquier v\'ertice puede ser vecino a otro (no considerando, por supuesto, par\'ametros extra
  que pudieran optimizar este proceso) y esto ocasionar\'ia un mayor tiempo de ejecuci\'on.\\

  Los resultados me hacen pensar que la concepci\'on original del AIG fue seguida y que, por lo tanto,
  su dise\~no  y adaptaci\'on a los t\'erminos del problema han sido los adecuados.

  \bibliography{ref}{}
  \bibliographystyle{plain}
\end{document}